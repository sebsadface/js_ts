%%%%%%%%%%%%%%%%%%%%% PACKAGE IMPORTS %%%%%%%%%%%%%%%%%%%%%
\documentclass[11pt]{article}
\usepackage{amsmath, amsfonts, amsthm, amssymb}
\usepackage{lmodern}
\usepackage{microtype}
\usepackage{fullpage}       
\usepackage{changepage}
\usepackage{hyperref}
\usepackage{blindtext}
\hypersetup{
    colorlinks=true,
    linkcolor=blue,
    filecolor=magenta,      
    urlcolor=blue,
    pdftitle={Overleaf Example},
    pdfpagemode=FullScreen,
    }
\urlstyle{same}

\newenvironment{level}%
{\addtolength{\itemindent}{2em}}%
{\addtolength{\itemindent}{-2em}}

\usepackage{amsmath,amsthm,amssymb}


\usepackage[x11names, rgb]{xcolor}
\usepackage{graphicx}
\usepackage[nooldvoltagedirection]{circuitikz}
\usetikzlibrary{decorations,arrows,shapes}

\usepackage{datetime}
\usepackage{etoolbox}
\usepackage{enumerate}
\usepackage{enumitem}
\usepackage{listings}
\usepackage{array}
\usepackage{varwidth}
\usepackage{tcolorbox}
\usepackage{circuitikz}
\usepackage{verbatim}
\usepackage[linguistics]{forest}
\usepackage{listings}
\usepackage{xcolor}

\newcommand\doubleplus{+\kern-1.3ex+\kern0.8ex}
\newcommand\mdoubleplus{\ensuremath{\mathbin{+\mkern-10mu+}}}

\definecolor{codegreen}{rgb}{0,0.6,0}
\definecolor{codegray}{rgb}{0.5,0.5,0.5}
\definecolor{codepurple}{rgb}{0.58,0,0.82}
\definecolor{backcolour}{rgb}{0.95,0.95,0.92}

\lstdefinelanguage{JavaScript}{
  keywords={typeof, new, true, false, catch, function, return, null, catch, switch, var, if, in, while, do, else, case, break},
  keywordstyle=\color{blue}\bfseries,
  ndkeywords={class, export, boolean, throw, implements, import, this},
  ndkeywordstyle=\color{darkgray}\bfseries,
  identifierstyle=\color{black},
  sensitive=false,
  comment=[l]{//},
  morecomment=[s]{/*}{*/},
  commentstyle=\color{purple}\ttfamily,
  stringstyle=\color{red}\ttfamily,
  morestring=[b]',
  morestring=[b]"
}

\lstdefinestyle{mystyle}{
    language=JavaScript,
    backgroundcolor=\color{backcolour},   
    commentstyle=\color{codegreen},
    keywordstyle=\color{magenta},
    numberstyle=\tiny\color{codegray},
    stringstyle=\color{codepurple},
    basicstyle=\ttfamily\footnotesize,
    breakatwhitespace=false,         
    breaklines=true,                 
    captionpos=b,                    
    keepspaces=true,                 
    numbers=left,                    
    numbersep=5pt,                  
    showspaces=false,                
    showstringspaces=false,
    showtabs=false,                  
    tabsize=2
}

\lstset{style=mystyle}
%%%%%%%%%%%%%%%%%%%%%%%% QUESTION # %%%%%%%%%%%%%%%%%%%%%%%%
%% You can ignore this for the most part. Basically it    %%
%% helps number your questions and creates a new page     %%
%% with each question for the aesthetics. It also creates %%
%% parts i.e. (a) (b) (c) for multiple part questions.    %%
%% To use do: \begin{question} ... \end{question}         %%
%% and: \begin{part} ... \end{part}                       %%
%%%%%%%%%%%%%%%%%%%%%%%%%%%%%%%%%%%%%%%%%%%%%%%%%%%%%%%%%%%%
\setlength{\parindent}{0pt}
\setlength{\parskip}{5pt plus 1pt}

\providetoggle{questionnumbers}
\settoggle{questionnumbers}{true}
\newcommand{\noquestionnumbers}{
    \settoggle{questionnumbers}{false}
}

\newcounter{questionCounter}
\newenvironment{question}[2][\arabic{questionCounter}]{%
    \ifnum\value{questionCounter}=0 \else {\newpage}\fi%
    \setcounter{partCounter}{0}%
    \vspace{.25in} \hrule \vspace{0.5em}%
    \noindent{\bf \iftoggle{questionnumbers}{Question #1: }{}#2}%
    \addtocounter{questionCounter}{1}%
    \vspace{0.8em} \hrule \vspace{.10in}%
}

\newcounter{partCounter}[questionCounter]
\renewenvironment{part}[1][\alph{partCounter}]{%
    \addtocounter{partCounter}{1}%
    \vspace{.10in}%
    \begin{indented}%
       {\bf (#1)} %
}{\end{indented}}

\def\indented#1{\list{}{}\item[]}
\let\indented=\endlist
\def\show#1{\ifdefempty{#1}{}{#1\\}}

%%%%%%%%%%%%%%%%%%%%%%%% SHORT CUTS %%%%%%%%%%%%%%%%%%%%%%%%
%% This is just to improve your quality of life. Instead  %%
%% of having to type long things, you can type short      %%
%% things. Ex: \IMP instead of \rightarrow to get ->      %%
%%%%%%%%%%%%%%%%%%%%%%%%%%%%%%%%%%%%%%%%%%%%%%%%%%%%%%%%%%%%
\def\IMP{\rightarrow}
\def\AND{\wedge}
\def\OR{\vee}
\def\BI{\leftrightarrow}
\def\DIFF{\setminus}
\def\SUB{\subseteq}

\newcolumntype{C}{>{\centering\arraybackslash}m{1.5cm}}
\renewcommand\qedsymbol{$\blacksquare$}

%%%%%%%%%%%%%%%%%%%%%%%% ANSWER BOX %%%%%%%%%%%%%%%%%%%%%%%%
%% This will improve the quality of life for your TA.     %%
%% Use \begin{answer} and \end{answer} to surround your   %%
%% answers so it will be easier to see. You can adjust    %%
%% the background and frame colors below as needed.       %%
%% Here is the manual to help: tinyurl.com/tcolorbox-man  %%
%%%%%%%%%%%%%%%%%%%%%%%%%%%%%%%%%%%%%%%%%%%%%%%%%%%%%%%%%%%%
\newtcolorbox{answer}
{
  colback   = green!5!white,    % Background color
  colframe  = green!75!black,   % Outline color
  box align = center,           % Align box on text line
  varwidth upper,               % Enables multi line input
  hbox                          % Bounds box to text width
}

%%%%%%%%%%%%%%%%% Identifying Information %%%%%%%%%%%%%%%%%
%% For 311, we'd rather you DIDN'T tell us who you are   %%
%% in your homework so that we're not biased when        %%
%% So, even if you fill this information in, it will not %%
%% show up in the document unless you uncomment \header  %%
%% below                                                 %%
%%%%%%%%%%%%%%%%%%%%%%%%%%%%%%%%%%%%%%%%%%%%%%%%%%%%%%%%%%%
\newcommand{\myhwname}{Homework 6}
\newcommand{\myname}{Sebastian Liu}
\newcommand{\myemail}{ll57@cs.washington.edu}
\newcommand{\mysection}{AD}
%%%%%%%%%%%%%%%%%%%%%%%%%%%%%%%%%%%%%%%%%%%%%%%%%%%%%%%%%%%

%%%%%%%%%%%%%%%%%%% Document Options %%%%%%%%%%%%%%%%%%%%%%
\noquestionnumbers
%%%%%%%%%%%%%%%%%%%%%%%%%%%%%%%%%%%%%%%%%%%%%%%%%%%%%%%%%%%

%%%%%%%%%%%%%%%%%%%%%%%% WORK BELOW %%%%%%%%%%%%%%%%%%%%%%%%
\begin{document}

\begin{center}
    \textbf{Homework 6 Written} \bigskip
\end{center}

%%%%%%%%%%%%%%%%%%%%%%%% Question# 2 %%%%%%%%%%%%%%%%%%%%%%%%
\begin{question}{2. Loops, I Did It Again (8 points)}
    \begin{part}
        \begin{lstlisting}[language=C++]
            let R: string[][] = [];
            let i: number = 0;
        //  {{P1: R = [] and i = 0}}
        //  {{Inv: R = replace(A[0 .. i - 1], M)}}
            while (i !== A.length) {
                const val = M.get(A[i]);
                if (val !== undefined) {
        //       {{P2: Inv and val = M[A[i]] and A[i] is in M and i != A.length}}
        //       {{Q2: R ++ val = replace(A[0..i], M)}}
                    R.push(val);
                } else {
        //       {{P3: Inv and val = M[A[i]] and A[i] is not M and i != A.length}}
        //       {{Q3: R ++ [A[i]] = replace(A[0..i], M)}}
                    R.push([A[i]]);
                }
                i = i + 1;
            }
        //  {{P4: R = replace(A[0 .. i - 1], M) and i = A.length}}
        //  {{Q4: R = replace(A, M)}}
        \end{lstlisting}
    \end{part}
\newpage
    \begin{part}
        \begin{answer}
            For Inv:
            \begin{align*}
                R &= [] \tag{given in $P_1$}\\
                &= \text{replace}([], M) \tag{Def of \text{replace}}\\
                &= \text{replace}(A[0 .. -1], M) \tag{since the range $0 .. -1$ is empty} \\
                &= \text{replace}(A[0 .. 0 - 1], M) \\
                &= \text{replace}(A[0 .. i - 1], M) \tag{since $i = 0$}
            \end{align*}
            For $Q_2$:
            \begin{align*}
                R \mdoubleplus \text{val} &= \text{replace}(A[0 .. i - 1], M) \mdoubleplus \text{val} \tag{since $R = \text{replace}(A[0 .. i - 1], M)$}\\
                &= \text{replace}(A[0 .. i - 1], M) \mdoubleplus M[A[i]] \tag{since val $= M[A[i]]$}\\
                &= \text{replace}(A[0 .. i - 1] \mdoubleplus A[i], M) \tag{Def of replace and since $A[i]$ is in $M$}\\
                &= \text{replace}(A[0 .. i], M) \tag{Def of $\mdoubleplus$ (i.e. $A[0 .. i - 1] \mdoubleplus A[i] = A[0 .. i]$)}
            \end{align*}
            Similarly, for $Q_3$:
            \begin{align*}
                R \mdoubleplus [A[i]] &= \text{replace}(A[0 .. i - 1], M) \mdoubleplus [A[i]] \tag{since $R = \text{replace}(A[0 .. i - 1], M)$}\\
                &= \text{replace}(A[0 .. i - 1] \mdoubleplus A[i], M) \tag{Def of replace and since $A[i]$ is in $M$}\\
                &= \text{replace}(A[0 .. i], M) \tag{Def of $\mdoubleplus$ (i.e. $A[0 .. i - 1] \mdoubleplus A[i] = A[0 .. i]$)}
            \end{align*}
            For $Q_4$:
            \begin{align*}
                R &= \text{replace}(A[0 .. i - 1], M) \tag{given in $P_4$}\\
                &= \text{replace}(A[0 .. A.\text{length} - 1], M) \tag{since $i = A.$length}\\
                &= \text{replace}(A, M) \tag{since $A[0 .. A.\text{length} - 1] = A$}
            \end{align*}
        \end{answer}
    \end{part}

\end{question}

%%%%%%%%%%%%%%%%%%%%%%%% Question# 3 %%%%%%%%%%%%%%%%%%%%%%%%
\begin{question}{3. Jumping Through Loops (12 points)}
    \begin{part}
        \begin{answer}
            Let $R$ be any arry of arrays, and let $P(A)$ be the claim "concat$(R \mdoubleplus [A]) = $concat$(R) \mdoubleplus A$ for any array $A$". \\
            We show $P(A)$ holds for any array $B \in A$ by structural induction.\\
            \textbf{Base Case ($A = []$):} 
            \begin{align*}
                \text{concat}(R \mdoubleplus [[]]) &= \text{concat}(R) \tag{Def of concat}\\
                &= \text{concat}(R) \mdoubleplus [] \tag{since $[]$ is an empty array}
            \end{align*}
            \textbf{Inductive Hypothesis:} Suppose $P(B)$ holds for an arbitrary array $B$ \\
            \textbf{Inductive Step ($P(B \mdoubleplus [w])$):} Let $w$ be an arbitrary element.
            \begin{align*}
                \text{concat}(R \mdoubleplus [B \mdoubleplus [w]]) &= \text{concat}(R \mdoubleplus [B]) \mdoubleplus [w] \tag{Def of concat} \\
                &= \text{concat}(R) \mdoubleplus B \mdoubleplus [w] \tag{Inductive Hypothesis} \\
                &= \text{concat}(R) \mdoubleplus (B \mdoubleplus [w])
            \end{align*}

            \textbf{Conclusion:} Therefore, $P(A)$ holds for any array $A$ by the principle of induction.
        \end{answer}
    \end{part}

    \begin{part}
            \begin{lstlisting}[language=C++]
                let S: string[] = [];
                let j: number = 0;
            //  {{P1: S = [] and j = 0}}
            //  {{Inv1: S = concat(R[0..j-1])}}
                while (j !== R.length) {
                    const A: string[] = R[j];
                    let k: number = 0;
            //      {{P2: Inv1, A = R[j], k = 0, and j !== R.length}}
            //      {{Inv2: S = concat(R[0..j-1]) ++ A[0..k-1]}}
                    while (k !== A.length) {
            //          {{P3: Inv2 and k !== A.length}}
            //          {{Q3: S ++ A[k] = concat(R[0..j-1]) ++ A[0..k]}}
                        S.push(A[k]);
                        k = k + 1;
                    }
            //      {{P4: S = concat(R[0..j-1]) ++ A[0..k-1] and k = A.length}}
            //      {{Q4: S = concat(R[0..j-1]) ++ A}}
                    j = j + 1;
                }
            //  {{P5: S = concat(R[0..j-1]) and j = R.length}}
            //  {{Q5: S = concat(R)}}
            \end{lstlisting}
    \end{part}
\newpage
    \begin{part}
        \begin{answer}
            For Inv1:
            \begin{align*}
                S &= [] \tag{given in $P_1$}\\
                &= \text{concat}([]) \tag{Def of \text{concat}}\\
                &= \text{concat}(R[0..-1]) \tag{since the range $0 .. -1$ is empty} \\
                &= \text{concat}(R[0 .. 0 - 1]) \\
                &= \text{concat}(R[0 .. j - 1]) \tag{since $j = 0$}
            \end{align*}
            For Inv2:
            \begin{align*}
                S &= \text{concat}(R[0..j-1]) \tag{given in $P_2$}\\
                &= \text{concat}(R[0..j-1] \mdoubleplus [[]]) \tag{Def of concat}\\
                &= \text{concat}(R[0..j-1] \mdoubleplus [A[0..-1]]) \tag{since the range $0 .. -1$ is empty}\\
                &= \text{concat}(R[0..j-1] \mdoubleplus [A[0..0-1]]) \\
                &= \text{concat}(R[0..j-1] \mdoubleplus [A[0..k-1]]) \tag{since $k = 0$}\\
                &= \text{concat}(R[0..j-1]) \mdoubleplus A[0..k-1] \tag{as the fact that was proven in part (a)}\\
            \end{align*}
            For $Q_3$:
            \begin{align*}
                S \mdoubleplus A[k] &=  \text{concat}(R[0..j-1]) \mdoubleplus A[0..k-1] \mdoubleplus A[k] \tag{since $S = \text{concat}(R[0..j-1]) \mdoubleplus A[0..k-1]$}\\
                &= \text{concat}(R[0..j-1]) \mdoubleplus (A[0..k-1] \mdoubleplus A[k]) \\
                &= \text{concat}(R[0..j-1]) \mdoubleplus A[0..k] \tag{Def of $\mdoubleplus$ (i.e. $A[0 .. k - 1] \mdoubleplus A[k] = A[0 .. k]$)}
            \end{align*}
            For $Q_4$:
            \begin{align*}
                S &= \text{concat}(R[0..j-1]) \mdoubleplus A[0..k-1] \tag{given in $P_4$}\\
                &= \text{concat}(R[0..j-1]) \mdoubleplus A[0..A.\text{length}-1] \tag{since $k = A.$length}\\
                &= \text{concat}(R[0..j-1]) \mdoubleplus A \tag{since $A[0 .. A.\text{length} - 1] = A$}
            \end{align*}
            For $Q_5$:
            \begin{align*}
                R &= \text{concat}(R[0..j-1]) \tag{given in $P_5$}\\
                &= \text{concat}(R[0..R.\text{length}-1]) \tag{since $j = R.$length}\\
                &= \text{concat}(R) \tag{since $R[0 .. R.\text{length} - 1] = R$}
            \end{align*}
        \end{answer}
    \end{part}
\end{question}

%%%%%%%%%%%%%%%%%%%%%%%% Question# 5 %%%%%%%%%%%%%%%%%%%%%%%%
\begin{question}{5. Words of a Feather Flock Together (20 points)}
    \begin{part}
            \begin{lstlisting}[language=C++]
                let k: number = -1;
        //      {{P1: k = -1}}
        //      {{Inv1: there is no index 0 <= j <= k such that sub[i] = all[j+i]
        //        for all i = 0..m-1}}
                while (k + sub.length !== all.length) {
                    k = k + 1;
                    let l: = number = 0;
        //          {{P2: k = k0 + 1, l = 0, k0 + m != n, and there is no index 
        //            0 <= j <= k0 such that sub[i] = all[j+i] for all 
        //            i = 0..m-1}}
        //          {{Inv2: there is no index 0 <= j <= k-1 such that 
        //            sub[i] = all[j+i] for all i = 0..m-1 and sub[i] = all[k+i] 
        //            for all i = 0..l-1}}
                    while (l !== sub.length && sub[l] === all[k + l]) {
        //              {{P3: sub[i] = all[k+i] for all i = 0..l-1, l != m, 
        //                and sub[l] = all[k+l]}}
        //              {{Q3: sub[i] = all[k+i] for all i = 0..l}}
                        l = l + 1;             
                    }
        //          {{P4: Inv2, and l = m or sub[l] != all[k+l]}}
                    if (l === sub.length) {
        //              {{P5: Inv2, and l = m}}
        //              {{Q5: sub[i] = all[k+i] for all i = 0..m-1}}
                        return true;
                    }
        //          {{P6: Inv2, sub[l] != all[k+l], and l != m}}
        //          {{Q6: Inv1}}
                }
        //      {{P7: Inv1 and k + m = n}}
        //      {{Q7: there is no index 0 <= j <= n-m such that sub[i] = all[j+i] 
        //        for all i = 0..m-1}}
                return false;
            \end{lstlisting}
    \end{part}

    \begin{part}
        \begin{answer}
            For Inv1:\\\\
             Setting $k = -1$ makes the range $ 0 \le j \le -1$, which is empty. Hence, Inv1 no longer makes any claims, so it is vacuously true.\\\\\\
            For Inv2: \\\\
            For the first part, since $k = k_0 + 1$, we know $k_0 = k - 1$. Since we know there is no index $0 \le j \le k_0$ such that sub[$i$] = all[$j+i$] for all $i = 0..m-1$, subsituting $k_0$
             with $k - 1$, we get the fact that there is no index $0 \le j \le k - 1$ such that sub[$i$] = all[$j+i$] for all $i = 0..m-1$\\\\
            For the second part, setting $l = 0$ makes the range $ i = 0..0-1$, which is empty. Hence, sub[$l$] = all[$k+i$] for all $i = 0..l-1$  no longer makes any claims, so it is vacuously true. \\\\\\
            
            For $Q_3$:\\\\
            We can combine the two facts that sub$[i] =$ all$[k+i]$ for all $i = 0..l-1$ and sub$[l] = $ all$[k + l]$ (i.e. sub$[i] = $ all$[k + i]$ when $i = l$) in $P_3$, we get 
            $Q_3$: sub$[i] =$ all$[k+i]$ for all $i = 0..l$\\\\\\
            
            For $Q_5$:
            \begin{align*}
                \text{sub}[i] &= \text{all}[k + i] \text{ for all } i = 0..l-1 \tag{given in $P_5$}\\
                &= \text{all}[k + i] \text{ for all } i = 0..m-1 \tag{since $l = m$}
            \end{align*}
            For $Q_6$:\\\\
            Since sub$[l] \ne$ all$[k+l]$, $l \ne m$, we know there is no index $k$ such that sub[$l$] = all[$k+l$] for $l \ne m$ $(l < m)$. Since we know there is no index $0 \le j \le k - 1$ such that sub[$i$] = all[$j+i$] for all $i = 0..m-1$,
            combining this with the previous fact, we know that there is no index $0 \le j \le k$ such that sub[$i$] = all[$j+i$] for all $i = 0..m-1$ \\\\\\
            For $Q_7$:\\\\
            Since we know, from $P_7$, that $k + m = n$, we get $k = n -m$. Subsituting $k$ with $n - m$ in the known fact that there is no index $0 \le j \le k$ such that sub[$i$] = all[$j+i$] for all $i = 0..m-1$,
            we get that there is no index $0 \le j \le n - m$ such that sub[$i$] = all[$j+i$] for all $i = 0..m-1$.
        \end{answer}
    \end{part}
\end{question}

%%%%%%%%%%%%%%%%%%%%%%%% Question# 7 %%%%%%%%%%%%%%%%%%%%%%%%
\begin{question}{7. Extra Credit: She Wore a Raspberry Array (0 points)}
    \begin{part}
        \begin{answer}
            
        \end{answer}
    \end{part}
\end{question}
\end{document}