%%%%%%%%%%%%%%%%%%%%% PACKAGE IMPORTS %%%%%%%%%%%%%%%%%%%%%
\documentclass[11pt]{article}
\usepackage{amsmath, amsfonts, amsthm, amssymb}
\usepackage{lmodern}
\usepackage{microtype}
\usepackage{fullpage}       
\usepackage{changepage}
\usepackage{hyperref}
\usepackage{blindtext}
\hypersetup{
    colorlinks=true,
    linkcolor=blue,
    filecolor=magenta,      
    urlcolor=blue,
    pdftitle={Overleaf Example},
    pdfpagemode=FullScreen,
    }
\urlstyle{same}

\newenvironment{level}%
{\addtolength{\itemindent}{2em}}%
{\addtolength{\itemindent}{-2em}}

\usepackage{amsmath,amsthm,amssymb}


\usepackage[x11names, rgb]{xcolor}
\usepackage{graphicx}
\usepackage[nooldvoltagedirection]{circuitikz}
\usetikzlibrary{decorations,arrows,shapes}

\usepackage{datetime}
\usepackage{etoolbox}
\usepackage{enumerate}
\usepackage{enumitem}
\usepackage{listings}
\usepackage{array}
\usepackage{varwidth}
\usepackage{tcolorbox}
\usepackage{circuitikz}
\usepackage{verbatim}
\usepackage[linguistics]{forest}
\usepackage{listings}
\usepackage{xcolor}

\newcommand\doubleplus{+\kern-1.3ex+\kern0.8ex}
\newcommand\mdoubleplus{\ensuremath{\mathbin{+\mkern-10mu+}}}

\definecolor{codegreen}{rgb}{0,0.6,0}
\definecolor{codegray}{rgb}{0.5,0.5,0.5}
\definecolor{codepurple}{rgb}{0.58,0,0.82}
\definecolor{backcolour}{rgb}{0.95,0.95,0.92}

\lstdefinelanguage{JavaScript}{
  keywords={typeof, new, true, false, catch, function, return, null, catch, switch, var, if, in, while, do, else, case, break},
  keywordstyle=\color{blue}\bfseries,
  ndkeywords={class, export, boolean, throw, implements, import, this},
  ndkeywordstyle=\color{darkgray}\bfseries,
  identifierstyle=\color{black},
  sensitive=false,
  comment=[l]{//},
  morecomment=[s]{/*}{*/},
  commentstyle=\color{purple}\ttfamily,
  stringstyle=\color{red}\ttfamily,
  morestring=[b]',
  morestring=[b]"
}

\lstdefinestyle{mystyle}{
    language=JavaScript,
    backgroundcolor=\color{backcolour},   
    commentstyle=\color{codegreen},
    keywordstyle=\color{magenta},
    numberstyle=\tiny\color{codegray},
    stringstyle=\color{codepurple},
    basicstyle=\ttfamily\footnotesize,
    breakatwhitespace=false,         
    breaklines=true,                 
    captionpos=b,                    
    keepspaces=true,                 
    numbers=left,                    
    numbersep=5pt,                  
    showspaces=false,                
    showstringspaces=false,
    showtabs=false,                  
    tabsize=2
}

\lstset{style=mystyle}
%%%%%%%%%%%%%%%%%%%%%%%% QUESTION # %%%%%%%%%%%%%%%%%%%%%%%%
%% You can ignore this for the most part. Basically it    %%
%% helps number your questions and creates a new page     %%
%% with each question for the aesthetics. It also creates %%
%% parts i.e. (a) (b) (c) for multiple part questions.    %%
%% To use do: \begin{question} ... \end{question}         %%
%% and: \begin{part} ... \end{part}                       %%
%%%%%%%%%%%%%%%%%%%%%%%%%%%%%%%%%%%%%%%%%%%%%%%%%%%%%%%%%%%%
\setlength{\parindent}{0pt}
\setlength{\parskip}{5pt plus 1pt}

\providetoggle{questionnumbers}
\settoggle{questionnumbers}{true}
\newcommand{\noquestionnumbers}{
    \settoggle{questionnumbers}{false}
}

\newcounter{questionCounter}
\newenvironment{question}[2][\arabic{questionCounter}]{%
    \ifnum\value{questionCounter}=0 \else {\newpage}\fi%
    \setcounter{partCounter}{0}%
    \vspace{.25in} \hrule \vspace{0.5em}%
    \noindent{\bf \iftoggle{questionnumbers}{Question #1: }{}#2}%
    \addtocounter{questionCounter}{1}%
    \vspace{0.8em} \hrule \vspace{.10in}%
}

\newcounter{partCounter}[questionCounter]
\renewenvironment{part}[1][\alph{partCounter}]{%
    \addtocounter{partCounter}{1}%
    \vspace{.10in}%
    \begin{indented}%
       {\bf (#1)} %
}{\end{indented}}

\def\indented#1{\list{}{}\item[]}
\let\indented=\endlist
\def\show#1{\ifdefempty{#1}{}{#1\\}}

%%%%%%%%%%%%%%%%%%%%%%%% SHORT CUTS %%%%%%%%%%%%%%%%%%%%%%%%
%% This is just to improve your quality of life. Instead  %%
%% of having to type long things, you can type short      %%
%% things. Ex: \IMP instead of \rightarrow to get ->      %%
%%%%%%%%%%%%%%%%%%%%%%%%%%%%%%%%%%%%%%%%%%%%%%%%%%%%%%%%%%%%
\def\IMP{\rightarrow}
\def\AND{\wedge}
\def\OR{\vee}
\def\BI{\leftrightarrow}
\def\DIFF{\setminus}
\def\SUB{\subseteq}

\newcolumntype{C}{>{\centering\arraybackslash}m{1.5cm}}
\renewcommand\qedsymbol{$\blacksquare$}

%%%%%%%%%%%%%%%%%%%%%%%% ANSWER BOX %%%%%%%%%%%%%%%%%%%%%%%%
%% This will improve the quality of life for your TA.     %%
%% Use \begin{answer} and \end{answer} to surround your   %%
%% answers so it will be easier to see. You can adjust    %%
%% the background and frame colors below as needed.       %%
%% Here is the manual to help: tinyurl.com/tcolorbox-man  %%
%%%%%%%%%%%%%%%%%%%%%%%%%%%%%%%%%%%%%%%%%%%%%%%%%%%%%%%%%%%%
\newtcolorbox{answer}
{
  colback   = green!5!white,    % Background color
  colframe  = green!75!black,   % Outline color
  box align = center,           % Align box on text line
  varwidth upper,               % Enables multi line input
  hbox                          % Bounds box to text width
}

%%%%%%%%%%%%%%%%% Identifying Information %%%%%%%%%%%%%%%%%
%% For 311, we'd rather you DIDN'T tell us who you are   %%
%% in your homework so that we're not biased when        %%
%% So, even if you fill this information in, it will not %%
%% show up in the document unless you uncomment \header  %%
%% below                                                 %%
%%%%%%%%%%%%%%%%%%%%%%%%%%%%%%%%%%%%%%%%%%%%%%%%%%%%%%%%%%%
\newcommand{\myhwname}{Homework 7}
\newcommand{\myname}{Sebastian Liu}
\newcommand{\myemail}{ll57@cs.washington.edu}
\newcommand{\mysection}{AD}
%%%%%%%%%%%%%%%%%%%%%%%%%%%%%%%%%%%%%%%%%%%%%%%%%%%%%%%%%%%

%%%%%%%%%%%%%%%%%%% Document Options %%%%%%%%%%%%%%%%%%%%%%
\noquestionnumbers
%%%%%%%%%%%%%%%%%%%%%%%%%%%%%%%%%%%%%%%%%%%%%%%%%%%%%%%%%%%

%%%%%%%%%%%%%%%%%%%%%%%% WORK BELOW %%%%%%%%%%%%%%%%%%%%%%%%
\begin{document}

\begin{center}
    \textbf{Homework 7 Written} \bigskip
\end{center}

%%%%%%%%%%%%%%%%%%%%%%%% Question# 2 %%%%%%%%%%%%%%%%%%%%%%%%
\begin{question}{2. From Loop to Nuts (24 points)}
    \begin{part}
        \begin{lstlisting}[language=C++]
            let i: number = 0;
            let j: number = 0;
            let k: number = 0;
        //  {{P1: i = 0, j = 0, and k = 0}}
        //  {{Inv: L[0..k-1] = without(L0[0..i-1], R) and 
        //         L[k..n-1] = L0[k..n-1] and R[j-1] < L[i]}}
            while (i !== L.length) {
                if ((j === R.length) || (L[i] < R[j])) {
        //       {{P2: Inv, i != n, and (j = m or L[i] < R[j])}}
        //       {{Q2: L[0..k-1] ++ [L[i]] = without(L0[0..i], R) and 
        //             L[k+1..n-1] = L0[k+1..n-1] and R[j-1] < L[i+1]}}
                    L[k] = L[i];
                    i = i + 1;
                    k = k + 1;
                } else if (L[i] > R[j]) {
        //       {{P3: Inv, i != n, and (j != m and L[i] > R[j])}}
        //       {{Q3:  L[0..k-1] = without(L0[0..i-1], R) and 
        //              L[k..n-1] = L0[k..n-1] and R[j] < L[i]}}
                    j = j + 1;
                } else {
        //          {{P4: Inv, i != n, and (j != m and L[i] = R[j])}}
        //          {{Q4:  L[0..k-1] = without(L0[0..i], R) and 
        //                 L[k..n-1] = L0[k..n-1] and R[j] < L[i+1]}}
                    i = i + 1;
                    j = j + 1;  
                }
            }
        //  {{P5: Inv and i = n}}
        //  {{Q5: L[0..k-1] = without(L0, R)}}
        \end{lstlisting}
    \end{part}
\newpage
    \begin{part}
        \begin{answer}
            \textbf{For Inv:}
            \begin{align*}
                L[0..k-1] &= L[0..-1] \tag{since $k = 0$}\\
                &= [] \tag{since $L[0..-1]$ is empty}\\
                &= \text{without}([], R) \tag{Def of without}\\
                &= \text{without}(L_0[0..-1], R) \tag{since $L_0[0..-1]$ is empty}\\
                &= \text{without}(L_0[0..0-1], R) \\
                &= \text{without}(L_0[0..i-1], R) \tag{since $i = 0$}\\ \\
                L[k..n-1] &= L[0..n-1] \tag{since $k = 0$} \\
                &= L_0[0..n-1] \tag{by definition $L_0$ is the initial value of $L$} \\\\
                R[j-1] &= R[0-1] \tag{since $j = 0$}\\
                &= R[-1] \\
                &< L[i] \tag{since $R[-1]$ is undefined, this is vacuously true}
            \end{align*}
            \textbf{For $Q_2$:} \\
            Case 1 ($j = m$): 
            \begin{align*}
                 L[0..k-1] \mdoubleplus [L[i]] &= L[0..k-1] \mdoubleplus [L_0[i]] \tag{since $ 0 \le i \le n - 1$ and $L[k..n-1] = L_0[k..n-1]$ so $L[i] = L_0[i]$} \\
                 &= \text{without}(L_0[0..i-1], R) \mdoubleplus [L_0[i]] \tag{since $L[0..k-1] = \text{without}(L_0[0..i-1],R)$}\\
                 &= \text{without}(L_0[0..i-1], R[0..m-2] \mdoubleplus [R[m-1]]) \mdoubleplus [L_0[i]] \tag{Def of $\mdoubleplus$}\\
                 &= \text{without}(L_0[0..i-1], R[0..m-2] \mdoubleplus [R[j-1]]) \tag{since $j = m$}\\
                 &= \text{without}(L_0[0..i-1]\mdoubleplus [L_0[i]], R[0..m-2] \mdoubleplus [R[j-1]]) \tag{since $R[j-1] < L[i]$ and by the definition of without}\\
                 &= \text{without}(L_0[0..i], R[0..m-2] \mdoubleplus [R[j-1]]) \tag{Def of $\mdoubleplus$} \\
                 &= \text{without}(L_0[0..i], R[0..m-2] \mdoubleplus [R[m-1]]) \tag{since $j = m$}\\
                 &= \text{without}(L_0[0..i], R) \tag{Def of $\mdoubleplus$} \\\\
                 L[k+1..n-1] = L_0[k+1..n-1] \tag{since $L[k..n-1] = L_0[k..n-1]$} \\\\
                 R[j-1] < L[i+1] \tag{since $L$ is sorted and $R[j-1] < L[i]$}
            \end{align*}
            \end{answer}
        \end{part}
        \begin{part}[b continued]
            \begin{answer}
            Case 2 ($L[i] < R[j]$):
            \begin{align*}
                L[0..k-1] \mdoubleplus [L[i]] &= L[0..k-1] \mdoubleplus [L_0[i]] \tag{since $ 0 \le i \le n - 1$ and $L[k..n-1] = L_0[k..n-1]$ so $L[i] = L_0[i]$} \\
                &= \text{without}(L_0[0..i-1], R) \mdoubleplus [L_0[i]] \tag{since $L[0..k-1] = \text{without}(L_0[0..i-1],R)$}\\
                &= \text{without}(L_0[0..i-1], R[0..j-1]) \mdoubleplus [L_0[i]] \\ &\text{(since $L$ is sorted and $L[i] < R[j]$ so $L[i - 1] < R[j]$ and}\\ &\text{since the rest of $R$ does not matter)}\\
                &= \text{without}(L_0[0..i-1], R[0..j-1]) \mdoubleplus [L_0[i]]\\
                &= \text{without}(L_0[0..i-1], R[0..j-2] \mdoubleplus [R[j-1]]) \mdoubleplus [L_0[i]] \tag{Def of $\mdoubleplus$}\\
                &= \text{without}(L_0[0..i-1]\mdoubleplus [L_0[i]], R[0..j-2] \mdoubleplus [R[j-1]]) \tag{since $R[j-1] < L[i]$ and by the definition of without} \\
                &= \text{without}(L_0[0..i], R[0..j-1])\tag{Def of $\mdoubleplus$}\\
                &= \text{without}(L_0[0..i], R) \tag{since $L[i - 1] < R[j]$ and the rest of $R$ does not matter}\\\\
                L[k+1..n-1] = L_0[k+1..n-1] \tag{since $L[k..n-1] = L_0[k..n-1]$} \\\\
                R[j-1] < L[i+1] \tag{since $L$ is sorted and $R[j-1] < L[i]$}
           \end{align*}
            \textbf{For $Q_3$:}
            \begin{align*}
                L[0..k-1] &= \text{without}(L_0[0..i-1], R) \tag{as listed in $P_3$}\\\\
                L[k..n-1] &= L_0[k..n-1] \tag{as listed in $P_3$}\\\\
                R[j] &< L[i] \tag{since $L[i] > R[j]$}
            \end{align*}
        \end{answer}
    \end{part}
    \newpage
    \begin{part}[b continued]
        \begin{answer}
            \textbf{For $Q_4$:}
            \begin{align*}
                L[0..k-1] &= \text{without}(L_0[0..i-1], R) \tag{as listed in $P_4$}\\
                &= \text{without}(L_0[0..i-1], R[0..j-1]) \\ &\text{(since $ 0 \le i \le n - 1$ and $L[k..n-1] = L_0[k..n-1]$ so $L[i] = L_0[i]$ and } \\ &\text{since $L[i] = R[j]$ and $i < n$ and $L$ is sorted, so $L[i - 1] < R[j]$ therefore }\\ &\text{$L_0[i - 1] < R[j]$ and since the rest of $R$ does not matter)}\\
                &= \text{without}(L_0[0..i-1] \mdoubleplus [L_0[i]], R[0..j-1] \mdoubleplus [R[j]]) \tag{since $L[i] = L_0[i] = R[j]$ and by the definition of without}\\
                &= \text{without}(L_0[0..i], R[0..j]) \tag{Def of $\mdoubleplus$}\\
                &= \text{without}(L_0[0..i], R) \tag{since $L[i] = L_0[i] = R[j]$ and the rest of $R$ does not matter}\\\\
                L[k..n-1] &= L_0[k..n-1] \tag{as listed in $P_4$}\\\\
                R[j] < L[i+1] \tag{since $R[j-1] < L[i]$ and $L, R$ are sorted and only contain distinct elements}
            \end{align*}
            \textbf{For $Q_5$:}
            \begin{align*}
                L[0..k-1] &= \text{without}(L_0[0..i-1], R) \tag{as listed in $P_5$}\\
                &=\text{without}(L_0[0..n-1], R) \tag{since $i = n$}\\
                &=\text{without}(L_0, R) \tag{Def of $L_0$}
            \end{align*}
        \end{answer}
    \end{part}

\end{question}

%%%%%%%%%%%%%%%%%%%%%%%% Question# 3 %%%%%%%%%%%%%%%%%%%%%%%%
\begin{question}{3. Jumping Through Loops (12 points)}
    \begin{part}[f]
        \begin{answer}
            Despite it is simpler to implement and more performant, returning the internal array directly can expose the internal state of the SortedNumberSet, 
            which can potentially be modified from outside of the class. An external client could alter the array, potentially violating the representation 
            invariant of the SortedNumberSet, which can cause horrible hard-to-find bugs. 
        \end{answer}
    \end{part}

    \begin{part}[g]
        \begin{answer}
            Similar to part(f), if you return the internal List<number> directly, it would still expose the internal state of the SortedNumberSet object, 
            which could be modified externally, potentially violating the representation invariant and cause horrible and hard-to-find bugs. Therefore, 
            returning a copy or an immutable view of the internal structure would still be a better design choice.  
        \end{answer}
    \end{part}

\end{question}

%%%%%%%%%%%%%%%%%%%%%%%% Question# 5 %%%%%%%%%%%%%%%%%%%%%%%%
\begin{question}{6. Extra Credit: Size In the Back of Your Head (0 points)}
    
\end{question}

%%%%%%%%%%%%%%%%%%%%%%%% Question# 7 %%%%%%%%%%%%%%%%%%%%%%%%
\begin{question}{7. Extra Credit: Nothing to Contain About (0 points)}
   
\end{question}
\end{document}