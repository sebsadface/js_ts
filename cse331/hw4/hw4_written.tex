%%%%%%%%%%%%%%%%%%%%% PACKAGE IMPORTS %%%%%%%%%%%%%%%%%%%%%
\documentclass[11pt]{article}
\usepackage{amsmath, amsfonts, amsthm, amssymb}
\usepackage{lmodern}
\usepackage{microtype}
\usepackage{fullpage}       
\usepackage{changepage}
\usepackage{hyperref}
\usepackage{blindtext}
\hypersetup{
    colorlinks=true,
    linkcolor=blue,
    filecolor=magenta,      
    urlcolor=blue,
    pdftitle={Overleaf Example},
    pdfpagemode=FullScreen,
    }
\urlstyle{same}

\newenvironment{level}%
{\addtolength{\itemindent}{2em}}%
{\addtolength{\itemindent}{-2em}}

\usepackage{amsmath,amsthm,amssymb}


\usepackage[x11names, rgb]{xcolor}
\usepackage{graphicx}
\usepackage[nooldvoltagedirection]{circuitikz}
\usetikzlibrary{decorations,arrows,shapes}

\usepackage{datetime}
\usepackage{etoolbox}
\usepackage{enumerate}
\usepackage{enumitem}
\usepackage{listings}
\usepackage{array}
\usepackage{varwidth}
\usepackage{tcolorbox}
\usepackage{circuitikz}
\usepackage[linguistics]{forest}
\usepackage{listings}
\usepackage{xcolor}

\definecolor{codegreen}{rgb}{0,0.6,0}
\definecolor{codegray}{rgb}{0.5,0.5,0.5}
\definecolor{codepurple}{rgb}{0.58,0,0.82}
\definecolor{backcolour}{rgb}{0.95,0.95,0.92}

\lstdefinelanguage{JavaScript}{
  keywords={typeof, new, true, false, catch, function, return, null, catch, switch, var, if, in, while, do, else, case, break},
  keywordstyle=\color{blue}\bfseries,
  ndkeywords={class, export, boolean, throw, implements, import, this},
  ndkeywordstyle=\color{darkgray}\bfseries,
  identifierstyle=\color{black},
  sensitive=false,
  comment=[l]{//},
  morecomment=[s]{/*}{*/},
  commentstyle=\color{purple}\ttfamily,
  stringstyle=\color{red}\ttfamily,
  morestring=[b]',
  morestring=[b]"
}

\lstdefinestyle{mystyle}{
    language=JavaScript,
    backgroundcolor=\color{backcolour},   
    commentstyle=\color{codegreen},
    keywordstyle=\color{magenta},
    numberstyle=\tiny\color{codegray},
    stringstyle=\color{codepurple},
    basicstyle=\ttfamily\footnotesize,
    breakatwhitespace=false,         
    breaklines=true,                 
    captionpos=b,                    
    keepspaces=true,                 
    numbers=left,                    
    numbersep=5pt,                  
    showspaces=false,                
    showstringspaces=false,
    showtabs=false,                  
    tabsize=2
}

\lstset{style=mystyle}
%%%%%%%%%%%%%%%%%%%%%%%% QUESTION # %%%%%%%%%%%%%%%%%%%%%%%%
%% You can ignore this for the most part. Basically it    %%
%% helps number your questions and creates a new page     %%
%% with each question for the aesthetics. It also creates %%
%% parts i.e. (a) (b) (c) for multiple part questions.    %%
%% To use do: \begin{question} ... \end{question}         %%
%% and: \begin{part} ... \end{part}                       %%
%%%%%%%%%%%%%%%%%%%%%%%%%%%%%%%%%%%%%%%%%%%%%%%%%%%%%%%%%%%%
\setlength{\parindent}{0pt}
\setlength{\parskip}{5pt plus 1pt}

\providetoggle{questionnumbers}
\settoggle{questionnumbers}{true}
\newcommand{\noquestionnumbers}{
    \settoggle{questionnumbers}{false}
}

\newcounter{questionCounter}
\newenvironment{question}[2][\arabic{questionCounter}]{%
    \ifnum\value{questionCounter}=0 \else {\newpage}\fi%
    \setcounter{partCounter}{0}%
    \vspace{.25in} \hrule \vspace{0.5em}%
    \noindent{\bf \iftoggle{questionnumbers}{Question #1: }{}#2}%
    \addtocounter{questionCounter}{1}%
    \vspace{0.8em} \hrule \vspace{.10in}%
}

\newcounter{partCounter}[questionCounter]
\renewenvironment{part}[1][\alph{partCounter}]{%
    \addtocounter{partCounter}{1}%
    \vspace{.10in}%
    \begin{indented}%
       {\bf (#1)} %
}{\end{indented}}

\def\indented#1{\list{}{}\item[]}
\let\indented=\endlist
\def\show#1{\ifdefempty{#1}{}{#1\\}}

%%%%%%%%%%%%%%%%%%%%%%%% SHORT CUTS %%%%%%%%%%%%%%%%%%%%%%%%
%% This is just to improve your quality of life. Instead  %%
%% of having to type long things, you can type short      %%
%% things. Ex: \IMP instead of \rightarrow to get ->      %%
%%%%%%%%%%%%%%%%%%%%%%%%%%%%%%%%%%%%%%%%%%%%%%%%%%%%%%%%%%%%
\def\IMP{\rightarrow}
\def\AND{\wedge}
\def\OR{\vee}
\def\BI{\leftrightarrow}
\def\DIFF{\setminus}
\def\SUB{\subseteq}

\newcolumntype{C}{>{\centering\arraybackslash}m{1.5cm}}
\renewcommand\qedsymbol{$\blacksquare$}

%%%%%%%%%%%%%%%%%%%%%%%% ANSWER BOX %%%%%%%%%%%%%%%%%%%%%%%%
%% This will improve the quality of life for your TA.     %%
%% Use \begin{answer} and \end{answer} to surround your   %%
%% answers so it will be easier to see. You can adjust    %%
%% the background and frame colors below as needed.       %%
%% Here is the manual to help: tinyurl.com/tcolorbox-man  %%
%%%%%%%%%%%%%%%%%%%%%%%%%%%%%%%%%%%%%%%%%%%%%%%%%%%%%%%%%%%%
\newtcolorbox{answer}
{
  colback   = green!5!white,    % Background color
  colframe  = green!75!black,   % Outline color
  box align = center,           % Align box on text line
  varwidth upper,               % Enables multi line input
  hbox                          % Bounds box to text width
}

%%%%%%%%%%%%%%%%% Identifying Information %%%%%%%%%%%%%%%%%
%% For 311, we'd rather you DIDN'T tell us who you are   %%
%% in your homework so that we're not biased when        %%
%% So, even if you fill this information in, it will not %%
%% show up in the document unless you uncomment \header  %%
%% below                                                 %%
%%%%%%%%%%%%%%%%%%%%%%%%%%%%%%%%%%%%%%%%%%%%%%%%%%%%%%%%%%%
\newcommand{\myhwname}{Homework 4}
\newcommand{\myname}{Sebastian Liu}
\newcommand{\myemail}{ll57@cs.washington.edu}
\newcommand{\mysection}{AD}
%%%%%%%%%%%%%%%%%%%%%%%%%%%%%%%%%%%%%%%%%%%%%%%%%%%%%%%%%%%

%%%%%%%%%%%%%%%%%%% Document Options %%%%%%%%%%%%%%%%%%%%%%
\noquestionnumbers
%%%%%%%%%%%%%%%%%%%%%%%%%%%%%%%%%%%%%%%%%%%%%%%%%%%%%%%%%%%

%%%%%%%%%%%%%%%%%%%%%%%% WORK BELOW %%%%%%%%%%%%%%%%%%%%%%%%
\begin{document}

\begin{center}
    \textbf{Homework 4 Written} \bigskip
\end{center}

%%%%%%%%%%%%%%%%%%%%%%%% Question# 2 %%%%%%%%%%%%%%%%%%%%%%%%
\begin{question}{2. Not My Cup of Tree (20 points)}
    \begin{part}
        \begin{answer}
            \begin{align*}
                values(makeBst(nil)) &= values(empty) \tag{Def of $makeBst$} \\
                &= nil \tag{Def of $values$}
            \end{align*}
        \end{answer}
    \end{part}

    \begin{part}
        \begin{answer}
            Let $(P,S) = split(m, cons(a,L))$, $m = \lfloor len(cons(a,L))/2 \rfloor$, and $S = cons(b, R)$
            \begin{align*}
                values(makeBst(cons(a, L))) &= values(node(b, makeBst(P), makeBst(R))) \tag{Def of $makeBst$} \\
                &= concat(values(makeBst(P)), cons(b, values(makeBst(R)))) \tag{Def of $values$} \\
                &= concat(P, cons(b, values(makeBst(R)))) \tag{Inductive Hypothesis on $U = P$} \\
                &= concat(P, cons(b, R)) \tag{Inductive Hypothesis on $U = R$} \\
                &= concat(P, S) \tag{Since $S = cons(b, R)$} \\
                &= cons(a, L) \tag{Lemma 1, and since $(P,S) = split(m, cons(a,L))$}
            \end{align*}
        \end{answer}
    \end{part}

    \begin{part}
        \begin{answer}
            
        \end{answer}
    \end{part}

\end{question}

%%%%%%%%%%%%%%%%%%%%%%%% Question# 4 %%%%%%%%%%%%%%%%%%%%%%%%
\begin{question}{4. Many More Fish in the Tree (20 points)}
    \begin{part}
        \begin{answer}
            
        \end{answer}
    \end{part}

    \begin{part}
        \begin{answer}
            
        \end{answer}
    \end{part}
\end{question}

%%%%%%%%%%%%%%%%%%%%%%%% Question# 5 %%%%%%%%%%%%%%%%%%%%%%%%
\begin{question}{5. Chomping at the Split (20 points)}
    \begin{part}[b]
        \begin{answer}

        \end{answer}
    \end{part}
\end{question}

%%%%%%%%%%%%%%%%%%%%%%%% Question# 6 %%%%%%%%%%%%%%%%%%%%%%%%
\begin{question}{6. Extra Credit: Live Tree or Die (0 points)}
    \begin{part}
        \begin{answer}

        \end{answer}
    \end{part}

    \begin{part}
        \begin{answer}

        \end{answer}
    \end{part}

    \begin{part}
        \begin{answer}

        \end{answer}
    \end{part}

    \begin{part}
        \begin{answer}

        \end{answer}
    \end{part}

    \begin{part}
        \begin{answer}

        \end{answer}
    \end{part}

    \begin{part}
        \begin{answer}

        \end{answer}
    \end{part}

    \begin{part}
        \begin{answer}

        \end{answer}
    \end{part}
\end{question}
\end{document}