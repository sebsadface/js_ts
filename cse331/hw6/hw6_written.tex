%%%%%%%%%%%%%%%%%%%%% PACKAGE IMPORTS %%%%%%%%%%%%%%%%%%%%%
\documentclass[11pt]{article}
\usepackage{amsmath, amsfonts, amsthm, amssymb}
\usepackage{lmodern}
\usepackage{microtype}
\usepackage{fullpage}       
\usepackage{changepage}
\usepackage{hyperref}
\usepackage{blindtext}
\hypersetup{
    colorlinks=true,
    linkcolor=blue,
    filecolor=magenta,      
    urlcolor=blue,
    pdftitle={Overleaf Example},
    pdfpagemode=FullScreen,
    }
\urlstyle{same}

\newenvironment{level}%
{\addtolength{\itemindent}{2em}}%
{\addtolength{\itemindent}{-2em}}

\usepackage{amsmath,amsthm,amssymb}


\usepackage[x11names, rgb]{xcolor}
\usepackage{graphicx}
\usepackage[nooldvoltagedirection]{circuitikz}
\usetikzlibrary{decorations,arrows,shapes}

\usepackage{datetime}
\usepackage{etoolbox}
\usepackage{enumerate}
\usepackage{enumitem}
\usepackage{listings}
\usepackage{array}
\usepackage{varwidth}
\usepackage{tcolorbox}
\usepackage{circuitikz}
\usepackage[linguistics]{forest}
\usepackage{listings}
\usepackage{xcolor}

\definecolor{codegreen}{rgb}{0,0.6,0}
\definecolor{codegray}{rgb}{0.5,0.5,0.5}
\definecolor{codepurple}{rgb}{0.58,0,0.82}
\definecolor{backcolour}{rgb}{0.95,0.95,0.92}

\lstdefinelanguage{JavaScript}{
  keywords={typeof, new, true, false, catch, function, return, null, catch, switch, var, if, in, while, do, else, case, break},
  keywordstyle=\color{blue}\bfseries,
  ndkeywords={class, export, boolean, throw, implements, import, this},
  ndkeywordstyle=\color{darkgray}\bfseries,
  identifierstyle=\color{black},
  sensitive=false,
  comment=[l]{//},
  morecomment=[s]{/*}{*/},
  commentstyle=\color{purple}\ttfamily,
  stringstyle=\color{red}\ttfamily,
  morestring=[b]',
  morestring=[b]"
}

\lstdefinestyle{mystyle}{
    language=JavaScript,
    backgroundcolor=\color{backcolour},   
    commentstyle=\color{codegreen},
    keywordstyle=\color{magenta},
    numberstyle=\tiny\color{codegray},
    stringstyle=\color{codepurple},
    basicstyle=\ttfamily\footnotesize,
    breakatwhitespace=false,         
    breaklines=true,                 
    captionpos=b,                    
    keepspaces=true,                 
    numbers=left,                    
    numbersep=5pt,                  
    showspaces=false,                
    showstringspaces=false,
    showtabs=false,                  
    tabsize=2
}

\lstset{style=mystyle}
%%%%%%%%%%%%%%%%%%%%%%%% QUESTION # %%%%%%%%%%%%%%%%%%%%%%%%
%% You can ignore this for the most part. Basically it    %%
%% helps number your questions and creates a new page     %%
%% with each question for the aesthetics. It also creates %%
%% parts i.e. (a) (b) (c) for multiple part questions.    %%
%% To use do: \begin{question} ... \end{question}         %%
%% and: \begin{part} ... \end{part}                       %%
%%%%%%%%%%%%%%%%%%%%%%%%%%%%%%%%%%%%%%%%%%%%%%%%%%%%%%%%%%%%
\setlength{\parindent}{0pt}
\setlength{\parskip}{5pt plus 1pt}

\providetoggle{questionnumbers}
\settoggle{questionnumbers}{true}
\newcommand{\noquestionnumbers}{
    \settoggle{questionnumbers}{false}
}

\newcounter{questionCounter}
\newenvironment{question}[2][\arabic{questionCounter}]{%
    \ifnum\value{questionCounter}=0 \else {\newpage}\fi%
    \setcounter{partCounter}{0}%
    \vspace{.25in} \hrule \vspace{0.5em}%
    \noindent{\bf \iftoggle{questionnumbers}{Question #1: }{}#2}%
    \addtocounter{questionCounter}{1}%
    \vspace{0.8em} \hrule \vspace{.10in}%
}

\newcounter{partCounter}[questionCounter]
\renewenvironment{part}[1][\alph{partCounter}]{%
    \addtocounter{partCounter}{1}%
    \vspace{.10in}%
    \begin{indented}%
       {\bf (#1)} %
}{\end{indented}}

\def\indented#1{\list{}{}\item[]}
\let\indented=\endlist
\def\show#1{\ifdefempty{#1}{}{#1\\}}

%%%%%%%%%%%%%%%%%%%%%%%% SHORT CUTS %%%%%%%%%%%%%%%%%%%%%%%%
%% This is just to improve your quality of life. Instead  %%
%% of having to type long things, you can type short      %%
%% things. Ex: \IMP instead of \rightarrow to get ->      %%
%%%%%%%%%%%%%%%%%%%%%%%%%%%%%%%%%%%%%%%%%%%%%%%%%%%%%%%%%%%%
\def\IMP{\rightarrow}
\def\AND{\wedge}
\def\OR{\vee}
\def\BI{\leftrightarrow}
\def\DIFF{\setminus}
\def\SUB{\subseteq}

\newcolumntype{C}{>{\centering\arraybackslash}m{1.5cm}}
\renewcommand\qedsymbol{$\blacksquare$}

%%%%%%%%%%%%%%%%%%%%%%%% ANSWER BOX %%%%%%%%%%%%%%%%%%%%%%%%
%% This will improve the quality of life for your TA.     %%
%% Use \begin{answer} and \end{answer} to surround your   %%
%% answers so it will be easier to see. You can adjust    %%
%% the background and frame colors below as needed.       %%
%% Here is the manual to help: tinyurl.com/tcolorbox-man  %%
%%%%%%%%%%%%%%%%%%%%%%%%%%%%%%%%%%%%%%%%%%%%%%%%%%%%%%%%%%%%
\newtcolorbox{answer}
{
  colback   = green!5!white,    % Background color
  colframe  = green!75!black,   % Outline color
  box align = center,           % Align box on text line
  varwidth upper,               % Enables multi line input
  hbox                          % Bounds box to text width
}

%%%%%%%%%%%%%%%%% Identifying Information %%%%%%%%%%%%%%%%%
%% For 311, we'd rather you DIDN'T tell us who you are   %%
%% in your homework so that we're not biased when        %%
%% So, even if you fill this information in, it will not %%
%% show up in the document unless you uncomment \header  %%
%% below                                                 %%
%%%%%%%%%%%%%%%%%%%%%%%%%%%%%%%%%%%%%%%%%%%%%%%%%%%%%%%%%%%
\newcommand{\myhwname}{Homework 5}
\newcommand{\myname}{Sebastian Liu}
\newcommand{\myemail}{ll57@cs.washington.edu}
\newcommand{\mysection}{AD}
%%%%%%%%%%%%%%%%%%%%%%%%%%%%%%%%%%%%%%%%%%%%%%%%%%%%%%%%%%%

%%%%%%%%%%%%%%%%%%% Document Options %%%%%%%%%%%%%%%%%%%%%%
\noquestionnumbers
%%%%%%%%%%%%%%%%%%%%%%%%%%%%%%%%%%%%%%%%%%%%%%%%%%%%%%%%%%%

%%%%%%%%%%%%%%%%%%%%%%%% WORK BELOW %%%%%%%%%%%%%%%%%%%%%%%%
\begin{document}

\begin{center}
    \textbf{Homework 5 Written} \bigskip
\end{center}

%%%%%%%%%%%%%%%%%%%%%%%% Question# 1 %%%%%%%%%%%%%%%%%%%%%%%%
\begin{question}{1. Hit the Road, Back (10 points)}
    \begin{part}
        \begin{answer}
            \begin{align*}
                &\{\{x > 0 \}\} \\
                & y = 3 * x + 5; \\
                &\{\{x > 0 \text{ and } y = 3x + 5\}\} \\
                & z = y - x; \\
                &\{\{x > 0 \text{ and } y = 3x + 5 \text{ and } z = y - x\}\} \\
                & z = 2 * z; \\
                &\{\{x > 0 \text{ and } y = 3x + 5 \text{ and } z = 2(y - x) \}\} 
            \end{align*}

    We can see that this last assertion implies the stated post condition $z \ge 14$ as follows:
            \begin{align*}
                z &= 2 \cdot (y - x) \\
                &= 2 \cdot (3x + 5 - x) \tag{since $y = 3x + 5$} \\
                &= 2 \cdot(2x + 5) \\
                &= 4x + 10 \\ 
                &\ge 4 \cdot 1 + 10 \tag {since $x > 0$ and $x$ is an integer which implies that $x \ge 1$} \\ 
                &= 4 + 10 \\
                &= 14
            \end{align*}

        \end{answer}
    \end{part}
\newpage
    \begin{part}
        \begin{answer}
            \begin{align*}
                &\{\{ 3x \ge u \text{ and } v \le 1 \}\} \\
                &\{\{ 3x + 1 \ge u + v \}\} \\
                &y = u + v; \\
                &\{\{ 3x + 1 \ge y \}\} \\
                &x = x * 3; \\
                &\{\{ x + 1 \ge y \}\} \\
                &z = x + 1; \\
                &\{\{ z \ge y\}\}
            \end{align*}

            We can see that $3x \ge u$ and $v \le 1$ implies $3x + 1 \ge u +v$ since: 
            \begin{align*}
                   1 & \ge v \tag{since $v \le 1$} \\
                   3x + 1 &\ge u +v \tag{since $3x \ge u$, and adding both sides of the inequality}
            \end{align*}
        \end{answer}
    \end{part}
\end{question}

%%%%%%%%%%%%%%%%%%%%%%%% Question# 2 %%%%%%%%%%%%%%%%%%%%%%%%
\begin{question}{2. Just a Working If (16 points)}
    \begin{part}
        \begin{answer}
            \begin{align*}
                &\{\{y \ge 0 \}\} \\
                &\text{if } (y == 0) \{\\
                &\;\;\;\; \{\{ P_1: y \ge 1 \text{ and } y = 0\}\}\\
                &\;\;\;\; \{\{ Q_1: 1 > y\}\} \\
                &\;\;\;\; x = 1; \\
                &\;\;\;\; \{\{ x > y \}\} \\
                &\} \text{ else } \{ \\
                &\;\;\;\; \{\{ P_2: y \ge 1 \text{ and } y \ne 0\}\}\\
                &\;\;\;\; \{\{ Q_2: 2y > y \}\}\\
                &\;\;\;\; x = 2 * y; \\
                &\;\;\;\; \{\{ x > y \}\} \\
                &\} \\
                &\{\{ x > y \}\}
            \end{align*}
            To see that $P_1$ implies $Q_1$, note that $y = 0$ implies that $1 > y$.\\
            To see that $P_2$ implies $Q_2$, we can see that since $y \ge 1$, we can divide both sides of $2y > y$ by $y$ and get $2 > 1$ which is always true.
        \end{answer}
    \end{part}
    \newpage
    \begin{part}
        \begin{answer}
            \begin{align*}
                &\{\{x \ge 1 \text{ and } y > 0 \}\} \\
                &\text{if } (y >= 5*x) \{\\
                &\;\;\;\; \{\{ P_1: x \ge 1 \text{ and } y > 0 \text{ and } y \ge 5x \}\}\\
                &\;\;\;\; \{\{ Q_1: 20 / y \le 4\}\} \\
                &\;\;\;\; x = 20 / y; \\
                &\;\;\;\; \{\{ x \le 4 \}\} \\
                &\} \text{ else } \{ \\
                &\;\;\;\; \{\{ P_2: x \ge 1 \text{ and } y > 0 \text{ and } y < 5x  \}\}\\
                &\;\;\;\; \{\{ Q_2: (y - x) / x \le 4 \}\}\\
                &\;\;\;\; x = (y - x) / x; \\
                &\;\;\;\; \{\{ x \le 4 \}\} \\
                &\} \\
                &\{\{ x \le 4 \}\}
            \end{align*}
            To see that $P_1$ implies $Q_1$, we can calculate: \\
            \begin{align*}
                20 / y & \le 20 / 5x \tag{since $y \ge 5x$} \\
                & \le 20 / 5 \tag{since $x \ge 1$}\\
                & = 4
            \end{align*}
            To see that $P_2$ implies $Q_2$, we can calculate: \\
            \begin{align*}
                (y - x)/x & < (5x - x)/x \tag{since $y < 5x$}\\
                & = 4x /x \\
                & = 4 \tag{since $x \ge 1$}
            \end{align*}
        \end{answer}
    \end{part}
\end{question}

%%%%%%%%%%%%%%%%%%%%%%%% Question# 3 %%%%%%%%%%%%%%%%%%%%%%%%
\begin{question}{3. Hula-Loop (10 points)}
    \begin{part}
        \begin{answer}
            At the top of the loop initially, we can see that since $n = n_0$ and $n_0 \ge 0$, we have $n \ge 0$. Also because we know $m = 0$, we have:
            \begin{align*}
                5m &=  5 \cdot 0 \tag{since $m = 0$} \\
                &= 0 \\
                &= n_0 - n_0 \\
                &= n_0 - n \tag{since $n = n_0$} \\
            \end{align*}
        \end{answer}
    \end{part}

    \begin{part}
        \begin{answer}
            After the loop we know and that $5m = n_0 - n$ and that $n \ge 0$. Thus, we can see: 
            \begin{align*}
                5m &= n_0 - n \tag{as noted above} \\
                    &\le n_0 - 0 \tag{since $n \ge 0$} \\
                    &= n_0
            \end{align*}
            Since we also know that $n < 5$, so we have:
            \begin{align*}
                n_0 &= 5m + n \tag{since $5m = n_0 - n$}\\
                &< 5m + 5 \tag{since $n < 5$}  \\
                &= 5(m + 1)
            \end{align*}
        \end{answer}
    \end{part}
\newpage
    \begin{part}
        \begin{answer}
            We can start by filling in the assertions as follows:
            \begin{align*}
                &\{\{ \text{Inv: } 5m = n_0 - n \text{ and } n \ge 0 \}\} \\
                & \text{while } (n >= 5) \{\\
                    &\;\;\;\; \{\{ P_1: 5m = n_0 - n \text{ and } n \ge 0 \text{ and } n \ge 5 \}\}\\
                    &\;\;\;\; \{\{ Q_1: 5(m + 1) = n_0 - (n - 5) \text{ and } (n - 5) \ge 0 \}\}\\
                &\;\;\;\; m = m + 1;\\
                    &\;\;\;\; \{\{ P_2: 5(m - 1) = n_0 - n \text{ and } n \ge 0 \text{ and } n \ge 5 \}\}\\
                    &\;\;\;\; \{\{ Q_2: 5m = n_0 - (n - 5) \text{ and } (n - 5) \ge 0 \}\}\\
                &\;\;\;\; n = n - 5;  \\
                &\;\;\;\; \{\{  5m = n_0 - n \text{ and } n \ge 0 \}\}\\
                &\} 
            \end{align*}
            We can see that $P1$ implies $Q1$ by the following calculations:
            \begin{align*}
                n - 5 &\ge 5 - 5 \tag{since $n \ge 5$} \\
                &= 0
            \end{align*}
            \begin{align*}
                5(m + 1) &= 5m + 5\\
                &= n_0 - n + 5 \tag{since $5m = n_0 - n$}\\
                &= n_0 - (n - 5)
            \end{align*}
            And we can see that P2 implies Q2 by these, very similar, calculations:
            \begin{align*}
                n - 5 &\ge 5 - 5 \tag{since $n \ge 5$} \\
                &= 0
            \end{align*}
            \begin{align*}
                5m &= 5(m - 1) + 5\\
                &= n_0 - n + 5 \tag{since $5(m - 1) = n_0 - n$}\\
                &= n_0 - (n - 5)
            \end{align*}
        \end{answer}
    \end{part}
\end{question}

%%%%%%%%%%%%%%%%%%%%%%%% Question# 5 %%%%%%%%%%%%%%%%%%%%%%%%
\begin{question}{5. Loop Dreams (16 points)}
    \begin{part}
        \begin{answer}
            At the top of the loop initially, we can see that $a = 0$ and $b = 0$, so we have:
            \begin{align*}
                \text{amount-greater}(L_0, x) &= \text{amount-greater}(L, x) \tag{since $L = L_0$} \\
                &= 0 + \text{amount-greater}(L, x) \\
                &= a + \text{amount-greater}(L, x) \tag{since $a = 0$}
            \end{align*}
            Similarly:
            \begin{align*}
                \text{amount-less}(L_0, x) &= \text{amount-less}(L, x) \tag{since $L = L_0$} \\
                &= 0 + \text{amount-less}(L, x) \\
                &= b + \text{amount-less}(L, x) \tag{since $b = 0$}
            \end{align*}
        \end{answer}
    \end{part}

    \begin{part}
        \begin{answer}
            After the loop we know that $L =$ nil, amount-greater($L_0, x) = a +$ amount-greater$(L,x)$, and amount-less($L_0, x) = b +$ amount-greater$(L,x)$. 
            Thus, we can see:
            \begin{align*}
                a &= \text{amount-greater}(L_0, x) -  \text{amount-greater}(L, x) \tag{since amount-greater($L_0, x) = a +$ amount-greater$(L,x)$} \\\\
                &= \text{amount-greater}(L_0, x) -  \text{amount-greater}(nil, x) \tag{since $L =$ nil} \\
                &= \text{amount-greater}(L_0, x) - 0 \tag{Def of amount-greater} \\
                &= \text{amount-greater}(L_0, x)
            \end{align*}
            Similarly:
            \begin{align*}
                b &= \text{amount-less}(L_0, x) -  \text{amount-less}(L, x) \tag{since amount-less($L_0, x) = b +$ amount-less$(L,x)$} \\\\
                &= \text{amount-less}(L_0, x) -  \text{amount-less}(nil, x) \tag{since $L =$ nil} \\
                &= \text{amount-less}(L_0, x) - 0 \tag{Def of amount-less} \\
                &= \text{amount-less}(L_0, x)
            \end{align*}
        \end{answer}
    \end{part}

    \newpage

    \begin{part}
        \begin{answer}
            We can start by filling in the assertions as follows:
            \begin{align*}
                &\{\{ \text{amount-greater}(L_0, x) = a + \text{amount-greater}(L,x) \text{ and } \\ 
                & \;\;\;\; \text{amount-less}(L_0, x) = b + \text{amount-less}(L,x) \text{ and } L \ne \text{nil} \}\} \\
                & \text{if } (\text{L.hd $>$ x}) \{\\\\
                    &\;\;\;\; \{\{ P_1: \text{amount-greater}(L_0, x) = a + \text{amount-greater}(L,x) \text{ and } \\ 
                    & \;\;\;\; \text{amount-less}(L_0, x) = b + \text{amount-less}(L,x) \text{ and } L.\text{hd} > x \}\}\\
                    &\;\;\;\; \{\{ Q_1: \text{amount-greater}(L_0, x) = a + (L.\text{hd} - x) + \text{amount-greater}(L.\text{tl},x) \text{ and } \\ 
                    & \;\;\;\; \text{amount-less}(L_0, x) = b + \text{amount-less}(L.\text{tl},x) \text{ and } L \ne \text{nil} \}\}\\\\
                &\;\;\;\; \text{a = a + (L.hd - x)};\\
                &\{\{ \text{amount-greater}(L_0, x) = a + \text{amount-greater}(L.\text{tl},x) \text{ and } \\ 
                & \;\;\;\; \text{amount-less}(L_0, x) = b + \text{amount-less}(L.\text{tl},x) \text{ and } L \ne \text{nil} \}\} \\
                &\} \text{ else if } (\text{L.hd $<$ x}) \{\\\\
                    &\;\;\;\; \{\{ P_2: \text{amount-greater}(L_0, x) = a + \text{amount-greater}(L,x) \text{ and } \\ 
                    & \;\;\;\; \text{amount-less}(L_0, x) = b + \text{amount-less}(L,x) \text{ and } L.\text{hd} < x\}\}\\
                    &\;\;\;\; \{\{ Q_2: \text{amount-greater}(L_0, x) = a + \text{amount-greater}(L.\text{tl},x) \text{ and } \\ 
                    & \;\;\;\; \text{amount-less}(L_0, x) = b + (x -L.\text{hd}) + \text{amount-less}(L.\text{tl},x) \text{ and } L \ne \text{nil} \}\}\\\\
                &\;\;\;\; \text{b = b + (x - L.hd)};  \\
                &\{\{ \text{amount-greater}(L_0, x) = a + \text{amount-greater}(L.\text{tl},x) \text{ and } \\ 
                & \;\;\;\; \text{amount-less}(L_0, x) = b + \text{amount-less}(L.\text{tl},x) \text{ and } L \ne \text{nil} \}\} \\
                &\} \\
                &\{\{ \text{amount-greater}(L_0, x) = a + \text{amount-greater}(L.\text{tl},x) \text{ and } \\ 
                & \;\;\;\; \text{amount-less}(L_0, x) = b + \text{amount-less}(L.\text{tl},x) \text{ and } L \ne \text{nil} \}\} \\
                & \text{L = L.tl};\\
                &\{\{ \text{amount-greater}(L_0, x) = a + \text{amount-greater}(L,x) \text{ and } \\ 
                & \;\;\;\; \text{amount-less}(L_0, x) = b + \text{amount-less}(L,x) \text{ and } L \ne \text{nil} \}\} 
            \end{align*}
        \end{answer}
\newpage
\begin{answer}
            We can see that $P_1$ implies $Q_1$ by the following three calculations:\\
             (1)\begin{align*}
                L.\text{hd} > x \rightarrow L \ne \text{nil} \tag{L.hd $> x$ holds implies L is a nonempty list}
            \end{align*}
            (2)
            \begin{align*}
                \text{amount-greater}(L_0, x) &= a + \text{amount-greater}(L, x) \tag{as noted in $P_1$} \\
                &= a + \text{amount-greater}(cons(L.\text{hd}, L.\text{tl}), x) \tag{Def of list, and since L.hd $> x$ implies L is nonempty} \\\\
                &= a + (L.\text{hd} - x) + \text{amount-greater}(L.\text{tl}, x) \tag{Def of amount-greater, and since L.hd $> x$}
            \end{align*}
            (3)
            \begin{align*}
                \text{amount-less}(L_0, x) &= b + \text{amount-less}(L, x) \tag{as noted in $P_1$}\\
                &= b + \text{amount-less}(cons(L.\text{hd}, L.\text{tl}), x) \tag{Def of list, and since L.hd $> x$ implies L is nonempty} \\\\
                &= b + \text{amount-less}(L.\text{tl}, x) \tag{Def of amount-less, and since L.hd $> x$}
            \end{align*} \\

            Similarly, we can see that $P_2$ implies $Q_2$ by the following three calculations:\\
             (1)\begin{align*}
                L.\text{hd} < x \rightarrow L \ne \text{nil} \tag{L.hd $< x$ holds implies L is a nonempty list}
            \end{align*}
            (2)
            \begin{align*}
                \text{amount-greater}(L_0, x) &= a + \text{amount-greater}(L, x) \tag{as noted in $P_2$} \\
                &= a + \text{amount-greater}(cons(L.\text{hd}, L.\text{tl}), x) \tag{Def of list, and since L.hd $< x$ implies L is nonempty} \\\\
                &= a + \text{amount-greater}(L.\text{tl}, x) \tag{Def of amount-greater, and since L.hd $< x$}
            \end{align*}
            (3)
            \begin{align*}
                \text{amount-less}(L_0, x) &= b + \text{amount-less}(L, x) \tag{as noted in $P_2$}\\
                &= b + \text{amount-less}(cons(L.\text{hd}, L.\text{tl}), x) \tag{Def of list, and since L.hd $< x$ implies L is nonempty} \\\\
                &= b + (x - L.\text{hd}) + \text{amount-less}(L.\text{tl}, x) \tag{Def of amount-less, and since L.hd $< x$}
            \end{align*}
        \end{answer}
    \end{part}
\end{question}

%%%%%%%%%%%%%%%%%%%%%%%% Question# 8 %%%%%%%%%%%%%%%%%%%%%%%%
\begin{question}{Extra Credit: Do Bears Loop in the Woods? (0 points)}
    \begin{part}
        \begin{answer}
            \{\{Inv: concat$(L, R)$ = cons($S.$hd, concat(L, S.tl))\}\}
        \end{answer}
    \end{part}

    \begin{part}
        \begin{answer}
            
        \end{answer}
    \end{part}

    \begin{part}
        \begin{answer}
            
        \end{answer}
    \end{part}

    \begin{part}
        \begin{answer}
            
        \end{answer}
    \end{part}
\end{question}
\end{document}